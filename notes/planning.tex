
\documentclass[12pt]{article}
 \usepackage{blkarray}
\usepackage[margin=1in]{geometry} 
\usepackage{amsmath,amsthm,amssymb,scrextend}
\usepackage{fancyhdr}
\pagestyle{fancy}
\usepackage{pdfpages}
\usepackage{listings}
\usepackage{tikz-cd}
\newcommand{\cont}{\subseteq}
\usepackage{tikz}
\usepackage{pgfplots}
\usepackage{amsmath}
\usepackage[mathscr]{euscript}
\let\euscr\mathscr \let\mathscr\relax% just so we can load this and rsfs
\usepackage[scr]{rsfso}
\usepackage{amsthm}
\usepackage{amssymb}
\usepackage{multicol}
\usepackage{enumitem}
\usepackage[colorlinks=true, pdfstartview=FitV, linkcolor=blue,
citecolor=blue, urlcolor=blue]{hyperref}

\DeclareMathOperator{\arcsec}{arcsec}
\DeclareMathOperator{\arccot}{arccot}
\DeclareMathOperator{\arccsc}{arccsc}
\newcommand{\var}{\text{Var}}
\newcommand{\ddx}{\frac{d}{dx}}
\newcommand{\dfdx}{\frac{df}{dx}}
\newcommand{\ddxp}[1]{\frac{d}{dx}\left( #1 \right)}
\newcommand{\dydx}{\frac{dy}{dx}}
\let\ds\displaystyle
\newcommand{\intx}[1]{\int #1 \, dx}
\newcommand{\intt}[1]{\int #1 \, dt}
\newcommand{\defint}[3]{\int_{#1}^{#2} #3 \, dx}
\newcommand{\imp}{\Rightarrow}
\newcommand{\un}{\cup}
\newcommand{\inter}{\cap}
\newcommand{\ps}{\mathscr{P}}
\newcommand{\set}[1]{\left\{ #1 \right\}}
\newtheorem*{sol}{Solution}
\newtheorem*{claim}{Claim}
\newtheorem{problem}{Problem}
\begin{document}
 
% EVERYTHING ABOVE THIS LINE IS JUST PREABLE, NO NEED TO MESS WITH IT.__________________________________________________________________________________________
%
\renewcommand{\arraystretch}{1.25}

\lhead{Pemrograman Fungsional}
\chead{Proyek Akhir}
\rhead{19 Oktober 2025}
\begin{center}
    \Large
\textbf{Perencanaan Proyek Akhir Pemrograman Fungsional}
\textbf{Sistem Simulasi Ekosistem-Evolusi}\\
Proyek Individu\\
Pandu Adjie Sukarno (2206026826)
\end{center}
\noindent Berikut adalah beberapa fitur yang ada di sistem simulasi ini
\section{Lingkungan}
\begin{itemize}
    \item foodBush
    \item fruit
    \item carcass
    \item water
    \item seasonState
    \item 
\end{itemize}
\section{Spesies}
\begin{itemize}
    \item hungerBar
    \item ThirstBar
    \item energyBar
    \item sightRange
\end{itemize}
\section{Reproduksi}
\section{Penyebaran Penyakit}
\section{Visualisasi Data}

% THE DOCUMENT IS ESSENTIALLY DONE AT THIS POINT, NO NEED TO EDIT ANYTHING BELOW THIS______________________________________________________________________________________________

 
\end{document}