\documentclass[12pt]{article}
 \usepackage{blkarray}
\usepackage[margin=1in]{geometry} 
\usepackage{amsmath,amsthm,amssymb,scrextend}
\usepackage{fancyhdr}
\pagestyle{fancy}
\usepackage{pdfpages}
\usepackage{listings}
\usepackage{tikz-cd}
\newcommand{\cont}{\subseteq}
\usepackage{tikz}
\usepackage{amsmath}
\usepackage[mathscr]{euscript}
\let\euscr\mathscr\let\mathscr\relax% just so we can load this and rsfs
\usepackage[scr]{rsfso}
\usepackage{amsthm}
\usepackage{amssymb}
\usepackage{multicol}
\usepackage{enumitem}
\usepackage[colorlinks=true, pdfstartview=FitV, linkcolor=blue,
citecolor=blue, urlcolor=blue]{hyperref}

\DeclareMathOperator{\arcsec}{arcsec}
\DeclareMathOperator{\arccot}{arccot}
\DeclareMathOperator{\arccsc}{arccsc}
\newcommand{\var}{\text{Var}}
\newcommand{\ddx}{\frac{d}{dx}}
\newcommand{\dfdx}{\frac{df}{dx}}
\newcommand{\ddxp}[1]{\frac{d}{dx}\left( #1 \right)}
\newcommand{\dydx}{\frac{dy}{dx}}
\let\ds\displaystyle
\newcommand{\intx}[1]{\int #1 \, dx}
\newcommand{\intt}[1]{\int #1 \, dt}
\newcommand{\defint}[3]{\int_{#1}^{#2} #3 \, dx}
\newcommand{\imp}{\Rightarrow}
\newcommand{\un}{\cup}
\newcommand{\inter}{\cap}
\newcommand{\ps}{\mathscr{P}}
\newcommand{\set}[1]{\left\{ #1 \right\}}
\newtheorem*{sol}{Solution}
\newtheorem*{claim}{Claim}
\newtheorem{problem}{Problem}
\begin{document}
 
% EVERYTHING ABOVE THIS LINE IS JUST PREABLE, NO NEED TO MESS WITH IT.__________________________________________________________________________________________
%
\renewcommand{\arraystretch}{1.25}

\lhead{Pemrograman Fungsional}
\chead{Proyek Akhir}
\rhead{Last Update: \color{red}20-08-2025}
\begin{center}
    \Large
\textbf{Perencanaan Proyek Akhir Pemrograman Fungsional}
\textbf{Model Simulasi Ekosistem Evolusi di Haskell}\\
Proyek Individu\\
Pandu Adjie Sukarno (2206026826)
\end{center}
\noindent Proyek ini terinspirasi dari beberapa video YouTube yang saya suka mengenai simulasi biologi. Berikut adalah tautannya: 
\begin{enumerate}
    \item \href{https://youtube.com/playlist?list=PLKortajF2dPBWMIS6KF4RLtQiG6KQrTdB&si=hqaZhZ-h92zkJozj}{Primer - Evolution Playlist}
    \item \href{https://youtu.be/f7vH2Li9KOw?si=k78nDUqfp5TPhn00}{Icoso - I Made an Evolution Simulator (with silly little guys)}
    \item \href{https://youtu.be/r_It_X7v-1E?si=YXpYhTIrnf76xXY_}{Sebastian Lague - Coding Adventure: Simulating an Ecosystem}
\end{enumerate}
Berikut adalah rencana awal untuk fitur-fitur yang ada di model simulasi yang saya buat.
\section{Waktu}
Model simulasi akan berjalan menggunakan satuan waktu diskrit. Dalam satu waktu, akan terjadi berbagai macam kejadian (sesuai dengan yang akan dijelaskan berikutnya), kemudian baru lanjut ke waktu berikutnya. 
\section{Lingkungan (\texttt{Environment})}
\texttt{Environment} merupakan dunia dari simulasi ini. \texttt{Environment} didefinisikan sebagai \textit{grid} berukuran $x\times y$. Setiap populasi spesies akan tersebar di dalam \textit{grid} ini dan tidak bisa melewati batas \textit{grid} (dunia tidak \textit{loop}). 

\noindent Setiap \textit{tiles} bisa berisi air (\texttt{water}) atau \textit{tiles} kosong (\texttt{empty}). \texttt{water} hanya di-\textit{generate} di awal program dijalankan dan \textit{tiles}-nya bersifat terkluster. Suatu \texttt{Environment} memiliki batas maksimal untuk banyak \textit{tiles} \texttt{water} yang bisa di-\textit{generate} yaitu \texttt{maxWaterGenerated} dalam persentase. Persentase yang dimaksud adalah persentase dari keseluruhan \textit{grid} yang di-\textit{generate} merupakan \texttt{water}.

\noindent Berikut semua konfigurasi awal untuk \texttt{Environment}:
\begin{itemize}
    \item \texttt{environmentSize::(Integer, Integer)}
    \item \texttt{maxWaterGenerated::Float}
    \item \texttt{environmentSeed::Integer}
\end{itemize}

\noindent Berikut semua data di \texttt{Environment} yang bersifat dinamik atau bisa berubah seiring berjalannya waktu
\begin{itemize}
    \item \texttt{time::Integer}
    \item \texttt{fruitBushCount::Integer}
    \item \texttt{animalCount::Integer}
\end{itemize}
\section{Sumber Daya (\texttt{Resources})}
Terdapat tiga jenis sumber daya yang bisa muncul di dalam lingkungan, yaitu buah (\texttt{fruit}), daging (\texttt{meat}), dan air (\texttt{water}). Air bisa muncul secara tak terbatas di seluruh \textit{tiles} \texttt{water}, sedangkan buah dan daging memiliki tempat khusus untuk muncul.

\subsection{Semak Buah (\texttt{fruitBush})}
\texttt{fruitBush} akan di-\textit{generate} secara acak diseluruh \textit{tiles} saat awal program berjalan dan setiap beberapa waktu berdasarkan probabilitas \texttt{probFruitBushGenerated}. Suatu \texttt{Environment} memiliki batas maksimal untuk banyaknya \texttt{fruitBush} yang bisa di-\textit{generate} yaitu suatu nilai \texttt{maxFruitBushGenerated} dalam integer.

\noindent Setiap \texttt{fruitbush} bisa menghasilkan \texttt{fruit} saat awal program dijalankan atau beberapa waktu setelah buah pada semak tersebut habis dimakan. Buah hanya akan di-\textit{generate} jika semak sedang dalam keadaan kosong. Jumlah buah yang muncul pada suatu semak adalah acak dengan rata-rata \texttt{meanFruitGenerated}.

\noindent Setiap \texttt{fruitbush} memiliki umur. Apabila semak sudah melewati batas umurnya, maka semak akan mati dan menghilang dari lingkungan beserta buah di dalamnya.

\noindent Berikut semua konfigurasi awal untuk semua \texttt{fruitBush}:
\begin{itemize}
    \item \texttt{maxFruitBushGenerated::Integer}
    \item \texttt{probFruitBushGenerated::Float }
    \item \texttt{avgFruitGenerated::Integer}
    \item \texttt{fruitBushLifeSpan::Integer}
    \item \texttt{fruitBushSeed::Integer}
\end{itemize}
\noindent Berikut semua data di satuan \texttt{fruitBush} yang bersifat dinamik atau bisa berubah seiring berjalannya waktu
\begin{itemize}
    \item \texttt{age::Integer}
    \item \texttt{gridPosition::(Integer, Integer)}
    \item \texttt{fruitCount::Integer}
\end{itemize}
\subsection{Mayat Hewan (\texttt{corpse})}
Setiap kali ada hewan yang mati, entah karena diburu atau kehabisan energi, \texttt{corpse} akan muncul. Ketika muncul, \texttt{meat} juga akan otomatis di-\textit{generate} secara acak dengan rata-rata banyaknya daging berdasarkan ukuran dari hewan tersebut.

\noindent Setiap \texttt{corpse} juga memiliki durasi sebelum membusuk. Apabila sudah melewati batas umurnya, maka \texttt{corpse} akan menghilang dari lingkungan beserta daging di dalamnya.
\noindent Berikut semua konfigurasi awal untuk semua \texttt{corpse}:
\begin{itemize}
    \item \texttt{corpseLifeSpan::Integer}
\end{itemize}
\noindent Berikut semua data di satuan \texttt{corpse} yang bersifat dinamik atau bisa berubah seiring berjalannya waktu
\begin{itemize}
    \item \texttt{age::Integer}
    \item \texttt{gridPosition::(Integer, Integer)}
    \item \texttt{meatCount::Integer}
\end{itemize}
\section{\texttt{Spesies}}
Sebelum simulasi dimulai, kita bisa mendefinisikan spesies yang akan di simulasikan. Setiap spesies memiliki stat dasar dan variansi yang nantinya menjadi mean dan variansi dari distribusi normal untuk stat setiap makhluk hidup yang merupakan bagian dari spesies tersebut. Berikut adalah stat yang memerlukan konfigurasi awal untuk setiap spesies: 
\begin{itemize}
    \item \texttt{spesiesId::Integer} -- ID dari spesies untuk mempermudah pengambilan data
    \item \texttt{spesiesName::String} -- Nama spesies
    \item \texttt{spesiesSeed::Integer} -- Seed untuk RNG
    \item \texttt{spesiesRole::Role\{Predator|Prey\}} -- Peran spesies dalam lingkungan, yaitu sebagai mangsa atau pemangsa
    \item \texttt{baseLearningRate::Double} -- Nilai rata-rata untuk menentukan seberapa cepat otak spesies belajar dan beradaptasi dari lingkungannya
    \item \texttt{varLearningRate::Double} -- variansi laju kecepatan otak makhluk hidup belajar
    \item \texttt{baseDiscountFactor::Double} -- Nilai rata-rata untuk mengontrol seberapa peduli spesies dengan \textit{reward} instan atau \textit{reward} jangka panjang
    \item \texttt{varDiscountFactor::Double} -- variansi faktor diskon 
    \item \texttt{baseLongevity::Integer} -- Rata-rata durasi kehidupan spesies dalam satuan waktu simulasi.
    \item \texttt{varLongevity::Integer} -- Variansi durasi kehidupan antar makhluk hidup dalam satu spesies
    \item \texttt{baseSpeed::Integer} -- Rata-rata kecepatan spesies dalam satuan \textit{grid} per waktu simulasi. Menggambarkan berapa banyak \textit{grid} yang bisa ditempuh spesies ketika bergerak dalam satu satuan waktu simulasi.
    \item \texttt{varSpeed::Integer} -- Variansi kecepatan spesies antar makhluk hidup dalam satu spesies
    \item \texttt{baseSenseRange::Integer} -- Rata-rata diameter rentang penglihatan spesies dalam satuan grid. Menggambarkan seberapa jauh makhluk hidup dapat mendeteksi sumber daya atau makhluk hidup lain di sekitarnya
    \item \texttt{varSenseRange::Integer} -- Variansi diameter rentang penglihatan antar makhluk hidup dalam satu spesies
    \item \texttt{baseMaxInventory::Integer} -- Rata-rata kapasitas maksimal penyimpanan makanan pribadi yang dimiliki makhluk hidup dalam satu spesies dalam satuan sumber daya
    \item \texttt{varMaxInventory::Integer} -- Variansi kapasitas maksimal penyimpanan makanan pribadi yang dimiliki makhluk hidup dalam satu spesies
    \item \texttt{baseSize::Double} -- Rata-rata ukuran spesies
    \item \texttt{varSize::Double} -- Variansi ukuran spesies
    \item \texttt{baseStrength::Double} -- Rata-rata kekuatan spesies
    \item \texttt{varStrength::Double} -- Variansi kekuatan spesies
    \item \texttt{baseEmpathy::Double} -- Rata-rata empati yang dimiliki spesies. Empati menunjukkan seberapa peduli suatu makhluk hidup dengan makhluk hidup lainnya dalam satu spesies
    \item \texttt{varEmpathy::Double} -- Variansi empati antar makhluk hidup dalam satu spesies

\end{itemize}
\section{Individu-Predator}
\begin{itemize}
    \item hungerBar
    \item ThirstBar
    \item \texttt{gender::Gender\{Male|Female\}}
    \item energyBar
    \item sightRange
    \item strength
\end{itemize}
\section{Individu-Prey}
\begin{itemize}
    \item hungerBar
    \item ThirstBar
    \item energyBar
    \item sightRange
\end{itemize}
\section{Reproduksi}
\section{Penyebaran Penyakit}
\section{Visualisasi Data}

% THE DOCUMENT IS ESSENTIALLY DONE AT THIS POINT, NO NEED TO EDIT ANYTHING BELOW THIS______________________________________________________________________________________________

 
\end{document}