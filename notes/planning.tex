\documentclass[12pt]{article}
 \usepackage{blkarray}
\usepackage[margin=1in]{geometry} 
\usepackage{amsmath,amsthm,amssymb,scrextend}
\usepackage{fancyhdr}
\pagestyle{fancy}
\usepackage{pdfpages}
\usepackage{listings}
\usepackage{tikz-cd}
\newcommand{\cont}{\subseteq}
\usepackage{tikz}
\usepackage{pgfplots}
\usepackage{amsmath}
\usepackage[mathscr]{euscript}
\let\euscr\mathscr\let\mathscr\relax% just so we can load this and rsfs
\usepackage[scr]{rsfso}
\usepackage{amsthm}
\usepackage{amssymb}
\usepackage{multicol}
\usepackage{enumitem}
\usepackage[colorlinks=true, pdfstartview=FitV, linkcolor=blue,
citecolor=blue, urlcolor=blue]{hyperref}

\DeclareMathOperator{\arcsec}{arcsec}
\DeclareMathOperator{\arccot}{arccot}
\DeclareMathOperator{\arccsc}{arccsc}
\newcommand{\var}{\text{Var}}
\newcommand{\ddx}{\frac{d}{dx}}
\newcommand{\dfdx}{\frac{df}{dx}}
\newcommand{\ddxp}[1]{\frac{d}{dx}\left( #1 \right)}
\newcommand{\dydx}{\frac{dy}{dx}}
\let\ds\displaystyle
\newcommand{\intx}[1]{\int #1 \, dx}
\newcommand{\intt}[1]{\int #1 \, dt}
\newcommand{\defint}[3]{\int_{#1}^{#2} #3 \, dx}
\newcommand{\imp}{\Rightarrow}
\newcommand{\un}{\cup}
\newcommand{\inter}{\cap}
\newcommand{\ps}{\mathscr{P}}
\newcommand{\set}[1]{\left\{ #1 \right\}}
\newtheorem*{sol}{Solution}
\newtheorem*{claim}{Claim}
\newtheorem{problem}{Problem}
\begin{document}
 
% EVERYTHING ABOVE THIS LINE IS JUST PREABLE, NO NEED TO MESS WITH IT.__________________________________________________________________________________________
%
\renewcommand{\arraystretch}{1.25}

\lhead{Pemrograman Fungsional}
\chead{Proyek Akhir}
\rhead{Last Update: \color{red}19-08-2025}
\begin{center}
    \Large
\textbf{Perencanaan Proyek Akhir Pemrograman Fungsional}
\textbf{Model Simulasi Ekosistem Evolusi di Haskell}\\
Proyek Individu\\
Pandu Adjie Sukarno (2206026826)
\end{center}
\noindent Proyek ini terinspirasi dari beberapa video YouTube yang saya suka mengenai simulasi biologi. 
\begin{enumerate}
    \item \href{https://youtube.com/playlist?list=PLKortajF2dPBWMIS6KF4RLtQiG6KQrTdB&si=hqaZhZ-h92zkJozj}{Primer - Evolution Playlist}
    \item \href{https://youtu.be/f7vH2Li9KOw?si=k78nDUqfp5TPhn00}{Icoso - I Made an Evolution Simulator (with silly little guys)}
    \item \href{https://youtu.be/r_It_X7v-1E?si=YXpYhTIrnf76xXY_}{Sebastian Lague - Coding Adventure: Simulating an Ecosystem}
\end{enumerate}
Berikut adalah fitur-fitur yang ada di model simulasi yang saya buat.
\section{Lingkungan (\texttt{Environment})}
\texttt{Environment} merupakan dunia dari simulasi ini. \texttt{Environment} didefinisikan sebagai \textit{grid} berukuran $x\times y$. Setiap populasi spesies akan tersebar di dalam \textit{grid} ini dan tidak bisa melewati batas \textit{grid} (dunia tidak \textit{loop}). 

\noindent Setiap \textit{grid} bisa berisi semak makanan (\texttt{foodBush}), air (\texttt{water}), atau \textit{grid} kosong (\texttt{empty}). \texttt{foodBush} akan di-\textit{generate} secara acak diseluruh \textit{grid} saat awal program berjalan dan setiap beberapa waktu. \texttt{water} hanya di-\textit{generate} di awal program dijalankan dan \textit{grid}-nya bersifat terkluster. 

\noindent Suatu \texttt{Environment} memiliki batas maksimal untuk \texttt{foodBush} dan \texttt{water} yang bisa di-\textit{generate}. Untuk \texttt{foodBush}, batas maksimalnya adalah \texttt{maxFoodBushGenerated} dalam integer. Sedangkan untuk \texttt{water}, batas maksimalnya adalah \texttt{maxWaterGenerated} dalam persentase, yaitu persentase dari keseluruhan \textit{grid} yang di-\textit{generate} merupakan \texttt{water}.

\begin{itemize}
    \item \texttt{sizeX::Integer}
    \item \texttt{sizeY::Integer}
    \item \texttt{MaxFoodBushGenerated::Integer}
    \item \texttt{MaxWaterGenerated::Float}
    \item 
\end{itemize}
\section{Sumber Daya}
\begin{itemize}
    \item fruit
    \item carcass
    \item water
\end{itemize}
\section{Spesies}
\begin{itemize}
    \item gender
    \item status 
    \item lifeSpan
\end{itemize}
\section{Individu-Predator}
\begin{itemize}
    \item hungerBar
    \item ThirstBar
    \item energyBar
    \item sightRange
    \item strength
\end{itemize}
\section{Individu-Prey}
\begin{itemize}
    \item hungerBar
    \item ThirstBar
    \item energyBar
    \item sightRange
\end{itemize}
\section{Reproduksi}
\section{Penyebaran Penyakit}
\section{Visualisasi Data}

% THE DOCUMENT IS ESSENTIALLY DONE AT THIS POINT, NO NEED TO EDIT ANYTHING BELOW THIS______________________________________________________________________________________________

 
\end{document}